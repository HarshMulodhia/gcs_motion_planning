\documentclass{article}
\usepackage[utf8]{inputenc}
\usepackage{amsmath}
\usepackage{graphicx}
\usepackage{hyperref}
\usepackage{geometry}
\usepackage{listings}
\usepackage{xcolor}
\usepackage{tcolorbox}
\usepackage{booktabs}
\usepackage{enumitem}

\geometry{a4paper, margin=1in}

\definecolor{codegreen}{rgb}{0,0.6,0}
\definecolor{codegray}{rgb}{0.5,0.5,0.5}
\definecolor{codepurple}{rgb}{0.58,0,0.82}
\definecolor{backcolour}{rgb}{0.95,0.95,0.92}

\lstdefinestyle{mystyle}{
    backgroundcolor=\color{backcolour},   
    commentstyle=\color{codegreen},
    keywordstyle=\color{magenta},
    numberstyle=\tiny\color{codegray},
    stringstyle=\color{codepurple},
    basicstyle=\footnotesize\ttfamily,
    breakatwhitespace=false,         
    breaklines=true,                 
    captionpos=b,                    
    keepspaces=true,                 
    numbers=left,                    
    numbersep=5pt,                  
    showspaces=false,                
    showstringspaces=false,
    showtabs=false,                  
    tabsize=2
}

\lstset{style=mystyle}

\title{File Documentation - GCS Motion Planning Framework}
\author{Harsh Mulodhia}
\date{December 10, 2024}

\begin{document}

\maketitle
\tableofcontents
\newpage

\section{Core Modules}

\subsection{\texttt{gcs\_builder.py}}
\textbf{Purpose}: Constructs the Graph of Convex Sets for motion planning problems.

\subsubsection{Class: \texttt{GCSBuilder}}
Main builder for GCS graphs.
\begin{description}
    \item[Attributes] \hfill \\
    \texttt{dimension: int} - Problem dimensionality.\
    \texttt{graph: nx.DiGraph} - NetworkX directed graph.\
    \texttt{convex\_sets: Dict} - Stores convex region data.
    \item[Key Methods] \hfill \\
    \texttt{add\_convex\_set(...)} - Adds a convex region to the graph.\
    \texttt{add\_edge(...)} - Creates a directed edge between regions.\
    \texttt{get\_shortest\_path()} - Finds the shortest path using Dijkstra's algorithm.
\end{description}

\subsubsection{Class: \texttt{ConvexSetBuilder}}
Provides static methods for creating standard convex regions like boxes and spheres.

\subsection{\texttt{solver.py}}
\textbf{Purpose}: Solves GCS optimization problems using convex optimization.

\subsubsection{Class: \texttt{GCSSolver}}
CVXPY-based trajectory optimization through convex regions.
\begin{description}
    \item[Attributes] \hfill \\
    \texttt{verbose: bool} - Toggles solver output.\
    \texttt{solver: str} - Backend solver (e.g., 'ECOS', 'SCS').
    \item[Key Methods] \hfill \\
    \texttt{solve(...)} - Solves for the optimal trajectory.
    \texttt{\_optimize\_trajectory(...)} - Internal method for formulating and solving the convex optimization problem.
\end{description}

\subsection{\texttt{agent.py}}
\textbf{Purpose}: Orchestrates the main training pipeline with monitoring and visualization.

\subsubsection{Class: \texttt{GCSTrainingAgent}}
Handles the main training loop, configuration, and integration of other modules.
\begin{description}
    \item[Key Methods] \hfill \\
    \texttt{train(...)} - Runs the main training loop.\
    \texttt{\_simulate\_step(...)} - Mocks a single training step (can be replaced with real logic).\
    \texttt{\_update\_visualization(...)} - Updates the 3D visualization.
    \texttt{\_save\_artifacts()} - Saves final outputs like dashboards.
\end{description}

\section{Visualization Modules}

\subsection{\texttt{meshcat\_visualizer.py}}
\textbf{Purpose}: Real-time, interactive 3D visualization via a web browser.
\begin{description}
    \item[Features] Trajectory rendering, obstacle visualization, and reference frame display.
\end{description}

\subsection{\texttt{pyvista\_visualizer.py}}
\textbf{Purpose}: Scientific-grade 3D rendering with advanced options.
\begin{description}
    \item[Features] Point cloud visualization, geometric primitives, and export to PNG/HTML.
\end{description}

\subsection{\texttt{plotly\_dashboard.py}}
\textbf{Purpose}: Interactive, web-based dashboard for training metrics.
\begin{description}
    \item[Features] Multi-metric tracking, moving averages, and statistical summaries.
\end{description}

\section{Training System}

\subsection{\texttt{training\_utils.py}}
\textbf{Purpose}: Provides utilities and stability mechanisms for robust training.
\begin{description}
    \item[Key Classes] \hfill \\
    \texttt{GCSTrainer} - Orchestrates training with features like gradient clipping and early stopping.\
    \texttt{Timer} - A context manager for timing code blocks.\
    \texttt{CheckpointManager} - Manages saving and loading of model checkpoints.\
    \texttt{GCSOptimizationCache} - An LRU cache for optimization results.
\end{description}

\subsection{\texttt{wandb\_integration.py}}
\textbf{Purpose}: Integrates with Weights \& Biases for experiment tracking.
\begin{description}
    \item[Features] Logging metrics, configurations, and artifacts to W\&B.
\end{description}

\section{Configuration \& Setup}
This section covers files related to project configuration and installation.

\subsection{\texttt{config.py}}
Manages default and custom configurations for the project.

\subsection{\texttt{training\_config.yaml}}
YAML file for setting hyperparameters and other training parameters.

\subsection{\texttt{setup.py}}
Standard Python file for package installation and dependency management.

\section{Notebooks}
This project includes four Jupyter notebooks that serve as tutorials and demonstrations.

\begin{itemize}
    \item \texttt{01\_gcs\_introduction.ipynb}: Introduces GCS theory and basic examples.
    \item \texttt{02\_visualization\_demo.ipynb}: Demonstrates advanced visualization techniques.
    \item \texttt{03\_training\_analysis.ipynb}: Analyzes training dynamics and stability.
    \item \texttt{04\_results\_presentation.ipynb}: Presents final results and performance metrics.
\end{itemize}

\section{Testing \& Scripts}
This section details the files used for testing and running the project.

\subsection{\texttt{test\_gcs\_planner.py}}
Unit tests for the core GCS planning module.

\subsection{\texttt{test\_visualizer.py}}
Tests for the visualization modules.

\subsection{\texttt{QUICK\_TEST.py}}
A script for quick, high-level verification of the project's functionality.

\subsection{\texttt{run\_training.sh}}
A shell script to execute the training pipeline.

\subsection{\texttt{install-dependencies.sh}}
A shell script for automated installation of project dependencies.

\end{document}
