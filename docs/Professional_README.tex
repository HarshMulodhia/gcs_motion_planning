\documentclass{article}
\usepackage[utf8]{inputenc}
\usepackage{amsmath}
\usepackage{graphicx}
\usepackage{hyperref}
\usepackage{geometry}
\usepackage{listings}
\usepackage{xcolor}
\usepackage{tcolorbox}
\usepackage{booktabs}

\geometry{a4paper, margin=1in}
\hypersetup{
    colorlinks=true,
    linkcolor=blue,
    filecolor=magenta,      
    urlcolor=cyan,
}

\definecolor{codegreen}{rgb}{0,0.6,0}
\definecolor{codegray}{rgb}{0.5,0.5,0.5}
\definecolor{codepurple}{rgb}{0.58,0,0.82}
\definecolor{backcolour}{rgb}{0.95,0.95,0.92}

\lstdefinestyle{mystyle}{
    backgroundcolor=\color{backcolour},   
    commentstyle=\color{codegreen},
    keywordstyle=\color{magenta},
    numberstyle=\tiny\color{codegray},
    stringstyle=\color{codepurple},
    basicstyle=\footnotesize\ttfamily,
    breakatwhitespace=false,         
    breaklines=true,                 
    captionpos=b,                    
    keepspaces=true,                 
    numbers=left,                    
    numbersep=5pt,                  
    showspaces=false,                
    showstringspaces=false,
    showtabs=false,                  
    tabsize=2
}

\lstset{style=mystyle}

\title{GCS Motion Planning: Graph of Convex Sets for Optimal Trajectory Planning}
\author{Harsh Mulodhia}
\date{December 10, 2024}

\begin{document}

\maketitle

\begin{abstract}
A comprehensive, production-ready framework for optimal motion planning using Graph of Convex Sets (GCS) with integrated visualization, advanced training utilities, and neural network integration capabilities.
\end{abstract}

\section{Key Features}
\begin{itemize}
    \item \textbf{Optimal Trajectory Planning}: Convex optimization-based motion planning with theoretical guarantees.
    \item \textbf{Collision Avoidance}: Guaranteed safety through convex region decomposition.
    \item \textbf{Advanced Visualization}: Real-time 3D rendering with MeshCat and scientific visualization with PyVista.
    \item \textbf{Interactive Dashboards}: Production-grade training metrics with Plotly.
    \item \textbf{Training Stability}: Includes gradient clipping, early stopping, and warm-start optimization.
\end{itemize}

\section{Quick Start}

\subsection{Installation}
\begin{lstlisting}[language=bash]
# Clone repository
git clone https://github.com/HarshMulodhia/gcs_motion_planning.git
cd gcs_motion_planning

# Install dependencies
bash install-dependencies.sh
\end{lstlisting}

\subsection{Basic Usage}
\begin{lstlisting}[language=Python]
from src.gcs_planner import GCSBuilder, GCSSolver, ConvexSetBuilder
import numpy as np

# 1. Create GCS graph
builder = GCSBuilder(dimension=3)

# 2. Define convex regions
start_region = ConvexSetBuilder.box(center=np.array([0,0,0]), half_lengths=np.array([0.5,0.5,0.5]))
builder.add_convex_set('start', start_region)
goal_region = ConvexSetBuilder.box(center=np.array([5,5,5]), half_lengths=np.array([0.5,0.5,0.5]))
builder.add_convex_set('goal', goal_region)

# 3. Connect regions
builder.add_edge('start', 'goal', weight=1.0)

# 4. Solve for optimal trajectory
solver = GCSSolver(verbose=True)
solution = solver.solve(builder, np.array([0,0,0]), np.array([5,5,5]))

if solution and solution['feasible']:
    print(f"Found {solution['length']} waypoints")
\end{lstlisting}

\section{Project Structure}
An overview of the main directories:
\begin{itemize}
    \item \texttt{src/}: Core source code for the GCS planner, solver, visualizers, and training utilities.
    \item \texttt{notebooks/}: Jupyter notebooks with tutorials and demonstrations.
    \item \texttt{tests/}: Unit tests for the framework.
    \item \texttt{configs/}: Configuration files, such as \texttt{training\_config.yaml}.
    \item \texttt{docs/}: Detailed documentation files.
\end{itemize}

\section{Documentation}
The project includes extensive documentation:
\begin{itemize}
    \item \textbf{ACADEMIC\_PAPER.md}: A full research paper detailing the theory and experiments.
    \item \textbf{FILE\_DOCUMENTATION.md}: A complete API reference for all modules.
    \item Four Jupyter notebooks that provide a hands-on introduction to the framework.
\end{itemize}

\section{Theoretical Foundations}
The framework is built on the GCS paradigm, which formulates motion planning as a mixed-integer convex program. This provides optimality guarantees and ensures collision-free paths by constraining the trajectory to lie within a sequence of convex, obstacle-free regions.

\section{Performance Metrics}
\begin{table}[h!]
\centering
\begin{tabular}{lccc}
\toprule
\textbf{Environment} & \textbf{Dimension} & \textbf{Success Rate} & \textbf{Plan Time (ms)} \\
\midrule
Grid World & 2 & 98\% & 45 \\
Narrow Passage & 3 & 95\% & 125 \\
High-Dimensional & 5 & 92\% & 285 \\
\bottomrule
\end{tabular}
\caption{Benchmark Performance Results}
\label{tab:performance_readme}
\end{table}

\section{Contributing}
Contributions are welcome. Please fork the repository, create a feature branch, and open a pull request. Ensure your code follows PEP 8 guidelines and includes appropriate documentation and tests.

\section{License}
This project is licensed under the MIT License. See the LICENSE file for details.

\section{Citation}
If you use this framework in your research, please use the following BibTeX entry:
\begin{lstlisting}[language=tex]
@software{mulodhia2024gcs,
  title={GCS Motion Planning: A Production-Ready Framework for Optimal Trajectory Planning},
  author={Mulodhia, Harsh},
  year={2024},
  url={https://github.com/HarshMulodhia/gcs_motion_planning}
}
\end{lstlisting}

\end{document}
