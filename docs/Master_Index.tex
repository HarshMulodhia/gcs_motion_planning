\documentclass{article}
\usepackage[utf8]{inputenc}
\usepackage{amsmath}
\usepackage{graphicx}
\usepackage{geometry}
\usepackage{listings}
\usepackage{xcolor}
\usepackage{tcolorbox}
\usepackage{booktabs}

\geometry{a4paper, margin=1in}

\title{Master Index - GCS Motion Planning Documentation Package}
\author{Harsh Mulodhia}
\date{December 10, 2024}

\begin{document}

\maketitle

\section{Quick Navigation}
\begin{description}
    \item[For Quick Overview:] Start with \texttt{README\_PROFESSIONAL.md}.
    \item[For Researchers:] Prioritize \texttt{ACADEMIC\_PAPER.md}.
    \item[For Developers:] Prioritize \texttt{FILE\_DOCUMENTATION.md}.
    \item[For Project Integration:] Reference \texttt{DELIVERABLES\_SUMMARY.md}.
\end{description}

\section{Complete Documentation List}

\subsection{\texttt{README\_PROFESSIONAL.md}}
An enhanced, professional README for the GitHub repository, featuring a quick start guide, system architecture, performance benchmarks, and contribution guidelines.

\subsection{\texttt{ACADEMIC\_PAPER.md}}
An 8,000+ word research-grade paper detailing the theory, implementation, and experimental results of the GCS framework. Includes mathematical formulations, algorithms, and performance analysis.

\subsection{\texttt{FILE\_DOCUMENTATION.md}}
A complete API reference for all 20 files in the project, providing detailed descriptions of classes, methods, parameters, and usage examples for each module.

\subsection{\texttt{DELIVERABLES\_SUMMARY.md}}
An integration guide that summarizes all created deliverables, provides quality metrics, and outlines recommended steps for updating the GitHub repository.

\section{Content Organization Map}
A visual map of the documentation structure is provided to help users navigate the content, from high-level READMEs to deep implementation details in the file documentation and theoretical concepts in the academic paper.

\section{How to Use This Documentation}
This section provides tailored reading paths for different audiences:
\begin{itemize}
    \item Researchers and Academics
    \item Software Developers
    \item DevOps / ML Engineers
    \item Students and Learners
\end{itemize}

\section{Finding Specific Information}
This section provides a quick lookup for finding information on specific topics, such as GCS theory, API usage, training configuration, performance benchmarks, and troubleshooting.

\section{Integration Timeline}
A suggested timeline for integrating the new documentation into the repository, broken down into immediate, short-term, and medium-term actions.

\section{Support Matrix}
A table that maps common questions to the best document for finding the answer, ensuring users can efficiently resolve their queries.

\end{document}
